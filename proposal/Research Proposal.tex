\documentclass[14pt, titlepage]{article}

\usepackage[margin=1in]{geometry}
\usepackage{graphicx}
\linespread{1.3}

\title{Sustainable Energy's Influence on the USA's GDP}
\author{Grant, Eduardo Gomez, Dylan Gouthro}
	
\begin{document}

\maketitle

	\section*{Abstract}
	
		There has been a debate on whether or not the country should adopt a more environmentally conscious energy resource and whether it’d be financially advantageous. Our focus in this research is to figure out if there is a correlation between increased implementation of sustainable energy and the growth in the U.S. economy. While our focus is on sustainable energy, we are going to be looking at both environmentally friendly sources and those that are harmful, as well as the gross domestic product (GDP). We will be using a linear regression model in order to test the significance that consumption of energy variables has on the GDP.  Our expectation is that when consumption of renewable energy is increased, there will also be an increase in the growth of the economy.
	
	
	\section*{Introduction}

		In the past century, humanity’s need for energy has seen an impressive growth. Our various tools, methods of transportation, and medical care all depend on a reliable source of electricity. The success of countries in the past century have been driven by the accumulation of energy sources. In previous years we have met these needs by burning fossil fuels since they have proved to be a cheap and steady supply of energy.\\ Recently, as a society we have come to realize that fossil fuels have a detrimental impact on our environment and continue to damage the ecosystem around us. Concerned citizens have voiced their objections, pointing to alternative ways of sustaining our populace. However the most pressing concern for our survival is the fact the fossil fuel is a finite source of energy. As we continue to drain sources of oil around the world, energy companies have invested more in sustainable energy sources such as solar, wind, hydro, and geothermal. \\ The rise of these markets provides an opportunity to increase the job market and decrease our reliance on foreign oil markets.  Critics of these emerging technologies argue that the overall cost of implementation is too great and would damage our economy by pouring money into a doomed product. We have collected data from the World Bank and the Energy Information Administration to analyze whether or not the increasing consumption of energy generated by renewable energy is associated with the growth in the United States Gross Domestic Product. The EIA provided us with annual energy consumption for the country dating back to 1960. From this we were able to narrow it down to the most significant sources of sustainable energy and have a record of the U.S's consumption in british thermal units. The correlation between GDP and sustainable technologies can help us determine if the decision to continue growth in these sectors is an economically beneficial one.
  


	
	\section*{Objectives}
		
			We are going to examine the relationships between individual energy resources, consisting of a mixture of sustainable and nonrenewable energy, against GDP. We wish to find out which specific types of energy have the strongest/weakest impact on the growth of the economy of our country.  
			
	\newpage			
	
	\section*{Variables}
	
	Variables used in this model:
	
		\begin{itemize}
		
			\item Hydroelectricity

			\item Wind energy
			
			\item Solar energy
			
			\item Geothermal energy
			
			\item Natural Gases
			
			\item Coal
			
			\item Fossil Fuels
			
			\item Nuclear Power

			\item Motor Gasoline
			
			\item Gross Domestic Product
						
		\end{itemize} \\ There will be three subcategories of variables that we are going to use. The sustainable resource variables will contain the energy consumption of hydroelectricity, wind energy, solar energy, geothermal, and natural gases. The nonrenewable resource variables that we will be examining will be the energy consumption of coal, fossil fuels, nuclear power, and motor gasoline. These two subcategories of variables will be our independent variables. As our dependent variable, we will be using Gross Domestic Product (GDP). These specific variables were chosen because, in their respective categories, they are the most promising resources that will have an impact on the economy of the United States.  Confounding variables and background variables are not known at this moment.  


	
	\section*{Questions and Hypotheses}

		\textit{Question}: Which type of renewable/non-renewable energy has the greatest/least impact on the growth of the GDP of the United States? \\ \\ \textit{Hypothesis}: Our expected hypothesis is that as energy consumption of sustainable energy resources increase, we will see a rise in the growth of the economy (GDP increases). 
		
	\section*{Methodology}

		\subsection*{Research Design}
			
			Instead of looking at the dollar amount of the GDP and the consumption of energy sources as a whole, we are going to look at the percent change in GDP per year, and the percent change of consumption of each individual energy source. \\ We will be taking data from a couple different sources. This shouldn’t change what the data means because nothing about the data is going to be misinterpreted since we are using the years as their common variable. \\ We will be using linear regression models to check the individual relationship between energy resources and the growth of the GDP as each year passes. Then, after we figure out which ones are the energy sources that are significantly affecting the change of the GDP, we will use a multiple linear regression model in order reveal which, of the higher ranking energy sources in terms of significance, has a higher positive correlation between each one and the GDP. The ultimate result of analyzing these models is to see whether the renewable energy sources have an impact on the GDP.
			
		\subsection*{Subject of Focus}
		
			\textit{Sample Size}: We will be focusing on the years 1970 to 2016 from the data set concerning the consumption of energy, renewable and nonrenewable, in the United States of America. We also have data from that same range of years on the GDP this country. 

		\subsection*{References}
		
			\begin{itemize}
								 	
		 		\item EIA - Independent Statistics and Analysis. \\ (Retrieved February 13, 2018, from https://www.eia.gov/opendata/qb.php?category=40236)
		 
		 		\item World Bank: \\ (Retrieved Feburary 16, 2018, \\ href{url}{http://databank.worldbank.org/data/reports.aspx?source=2\&series=NY.GDP.MKTP.CD\&country=}

			\end{itemize}

\end{document}